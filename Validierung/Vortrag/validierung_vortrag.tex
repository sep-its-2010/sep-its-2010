% $Header: /cvsroot/latex-beamer/latex-beamer/solutions/conference-talks/conference-ornate-20min.de.tex,v 1.7 2004/10/07 20:53:08 tantau Exp $

\documentclass[xcolor=dvipsnames]{beamer}

  	\usetheme{Frankfurt}
  	\setbeamertemplate{footline}[frame number]
	\definecolor{RoyalBlue3}{rgb}{0.23,0.37,0.8}
	\definecolor{RoyalBlue4}{rgb}{0.15,0.25,0.54}
	\definecolor{SlateGray4}{rgb}{0.42,0.48,0.54}
	\definecolor{SteelBlue4}{rgb}{0.21,0.39,0.54}
  	\usecolortheme[named=RoyalBlue3]{structure}
	\useinnertheme{circles}  


\usepackage{beamerthemesplit}
\usepackage{color}
\usepackage{graphicx}
\usepackage{epstopdf}
\usepackage{listings}
\usepackage[nomarkers]{pause}
\usepackage{hyperref}
        
\usepackage[german]{babel}
% oder was auch immer

\usepackage[utf8]{inputenc}
% oder was auch immer

\usepackage{times}
\usepackage[T1]{fontenc}


\title[Validierung] % (optional, nur bei langen Titeln ntig)
{e-puck Conquest}

\subtitle
{Validierung}

\author[Binder, Bürchner, Freund, Lorenz, Poxrucker, Wilhelm] % (optional, nur bei vielen Autoren)
{SEP - ITS 2010 \\ Max Binder \and Florian Bürchner \and Martin Freund
	\\ Florian Lorenz \and Andreas Poxrucker \and Andreas Wilhelm}
% - Namen mssen in derselben Reihenfolge wie im Papier erscheinen.
% - Der \inst{?} Befehl sollte nur verwendet werden, wenn die Autoren
%   unterschiedlichen Instituten angehren.

\institute[Universität Passau] % (optional, aber oft ntig)
{
  Fakultät für Informatik und Mathematik\\
  Universität Passau}

\AtBeginSubsection[]
{
  \begin{frame}<beamer>
    \frametitle{Inhaltsverzeichnis}
    \tableofcontents[currentsection,currentsubsection]
  \end{frame}
}


% Falls Aufzhlungen immer schrittweise gezeigt werden sollen, kann
% folgendes Kommando benutzt werden:

%\beamerdefaultoverlayspecification{<+->}

\begin{document}

\lstset{language=Java, basicstyle=\footnotesize, tabsize=2}

\begin{frame}
  \titlepage
\end{frame}

\begin{frame}
  \frametitle{Inhaltsverzeichnis}
  \tableofcontents
\end{frame}	

\section{Projektdaten}
		\begin{frame}
		\frametitle{Projektdaten}
   			\begin{itemize}
				\item[•]LOC Java (ohne Kommentare) 6656
				\item[•]Anzahl an Methoden 454
				\item[•]Anzahl an Packages 10
				\item[•]LOC Gesamt 24000
			\end{itemize}			
		\end{frame}

		
\section{Testvoraussetzungen}
		\begin{frame}
		\frametitle{Wie haben wir getestet?}
   			\begin{itemize}
				\item[•]Android JUnit 3 
				\item[•]HyperTerminal / Putty
				\item[•]GUI durch Benutzung
			\end{itemize}			
		\end{frame}

\section{Globale Testszenarien und Testfälle}
	\begin{frame}
	\frametitle{Globale Testszenarien und Testfälle}
		\begin{itemize}
			\item[•]Abweichung zum Pflichtenheft
			\begin{itemize}
				\item[/T100/]Steuerung der Fahrtgeschwindigkeit
				\item[/T160W/]Globale Lokalisierung
				\item[/T170W/]Zoomfunktion der Karte
            \end{itemize}
		\end{itemize}
	\end{frame}
	
		\begin{frame}
	\frametitle{Globale Testszenarien und Testfälle}
		\begin{itemize}
			\item[•]Alle anderen Testfälle wurden durchgeführt und bestanden
		
		            \begin{tabular}{l|p{2cm}}
				\hline
					\textbf{Testfall} & \textbf{bestanden}\\
				\hline \hline
					 /T50/ Kalibrierung & OK \\
 				\hline
					/T60/ Linienerkennung & OK \\
				\hline		
					/T70/ Bluetooth-Scan & OK \\
				\hline	
					/T80/ Broadcast-Test & OK \\
				\hline		
					/T90/ Knotenanalyse und manuelle Steuerung & OK \\
				\hline	
					/T110/ Steuerung per Beschleunigungssensor & OK \\
				\hline									
					/T120/ Erkundungstest & OK \\
				\hline			
					/T130/ Erweiterter Steuerungstest & OK \\
				\hline		
					/T140W/ Speichern der Kartendaten & OK \\
				\hline		
					/T150W/ Laden der Kartendaten & OK \\
				\hline		
			\end{tabular}
			\item[•]Weitere Testfälle		
		\end{itemize}
	\end{frame}

\section{Unit Tests}
	\begin{frame}
	\frametitle{Android JUnit}
		\begin{itemize}
			\item[•]GridMap
		\end{itemize}
				            \begin{tabular}{l|p{2cm}}
				\hline
					\textbf{Testmethode} & \textbf{bestanden}\\
				\hline \hline
					 testInsertNode() & OK \\
 				\hline
					testFrontierNodeRightT() & OK \\
				\hline		
					testFrontierNodeLeftT() & OK \\
				\hline		
					testFrontierNodeBottomT() & OK \\
				\hline	
					testFrontierNodeTopT() & OK \\
				\hline	
					testFrontierNodeCross() & OK \\
				\hline	
					testFrontierNodeBottomLeftEdge() & OK \\
				\hline	
					testFrontierNodeBottomRightEdge() & OK \\
				\hline	
					testFrontierNodeTopLeftEdge() & OK \\
				\hline	
					testFrontierNodeTopRightEdge() & OK \\
				\hline	
					testUpdateNode() & OK \\
				\hline	
					testMapBorders() & OK \\
				\hline	
					testSerializeMapInString() & OK \\
				\hline	
			\end{tabular}
	\end{frame}
	
	\begin{frame}
	\frametitle{Android JUnit}
		\begin{itemize}
			\item[•]ComManager
		\end{itemize}		
	    \begin{tabular}{l|p{2cm}}
				\hline
					\textbf{Testmethode} & \textbf{bestanden}\\
				\hline \hline
					 testAddClientAndSend() & OK \\
 				\hline
					testRemoveClient() & OK \\
				\hline		
			\end{tabular}
			\vspace{0.5cm}
		\begin{itemize}
				\item[•]Behaviour
		\end{itemize}	
		 	\begin{tabular}{l|p{2cm}}
				\hline
					\textbf{Testmethode} & \textbf{bestanden}\\
				\hline \hline
					 testExploreBehaviour() & OK \\
 				\hline		
			\end{tabular}
			\vspace{0.5cm}
		\begin{itemize}
				\item[•]AStarPathFinder
		\end{itemize}	
		 	\begin{tabular}{l|p{2cm}}
				\hline
					\textbf{Testmethode} & \textbf{bestanden}\\
				\hline \hline
					 testFindPuckMapNodeMapNodeArray() & OK \\
 				\hline		
			\end{tabular}	
	\end{frame}
	
	\begin{frame}
	\frametitle{Android JUnit}
		\begin{itemize}
			\item[•]Handler
		\end{itemize}
			\begin{tabular}{l|p{2cm}}
				\hline
					\textbf{Testmethode} & \textbf{bestanden}\\
				\hline \hline
					 testSimTurnHandler() & OK \\
 				\hline
					testSimStatusHandler() & OK \\
				\hline		
					testSimSpeedHandler() & OK \\
				\hline		
					testSimResetHandler() & OK \\
				\hline	
					testSimMoveHandler() & OK \\
				\hline	
					testSimLEDHandler() & OK \\
				\hline	
					testPuckStatusHandler() & OK \\
				\hline	
					testPuckRejectHandler() & OK \\
				\hline	
					testPuckOkHandler() & OK \\
				\hline	
					testPuckNodeHitHandler() & OK \\
				\hline	
					testPuckCollisionHandler() & OK \\
				\hline	
					testPuckAbyssHandler() & OK \\
				\hline
			\end{tabular}
	\end{frame}
	
	\begin{frame}
	\frametitle{Epuck Unit Test}
		\begin{itemize}
			\item[•]Ringpuffer
		\end{itemize}
			\begin{tabular}{l|p{2cm}}
				\hline
					\textbf{Testmethode} & \textbf{bestanden}\\
				\hline \hline
					 testRingPuffer() & OK \\
 				\hline
			\end{tabular}
	\end{frame}

		
\section{Herausforderungen und Probleme des Projekts}
	\begin{frame}
	\frametitle{Herausforderungen und Probleme des Projekts}
		\begin{itemize}
			\item[•]Knotenerkennung des E-puck Roboters
			\item[•]Kollisionserkennung der Roboter auf dem Spielfeld
			\item[•]RaceConditions aufgrund mehrerer Threads
		\end{itemize}
	\end{frame}
		

\section{Ende}
	\begin{frame}
		\begin{center}
			Vielen Dank für die Aufmerksamkeit!
		\end{center}
	\end{frame}
		
\end{document}


